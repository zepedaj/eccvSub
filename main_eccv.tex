\documentclass[runningheads]{llncs}

%\usepackage{times}
%\usepackage{epsfig}
\usepackage{graphicx}
\usepackage{caption}
%\usepackage{subfigure}
\usepackage{subcaption}
\usepackage{amsmath}
\usepackage{amssymb}
\usepackage{ruler}
\usepackage{color}
\usepackage[width=122mm,left=12mm,paperwidth=146mm,height=193mm,top=12mm,paperheight=217mm]{geometry}
%\newtheorem{proposition}{Proposition}
%\newtheorem{lemma}{Lemma}
%\newtheorem{proof}{Proof}
\newcommand{\RR}{\mathbb R}
\newcommand{\NNN}{\mathcal N}
\newcommand{\sumi}{\displaystyle{\sum_{i=1}^n}}

\begin{document}
% \renewcommand\thelinenumber{\color[rgb]{0.2,0.5,0.8}\normalfont\sffamily\scriptsize\arabic{linenumber}\color[rgb]{0,0,0}}
% \renewcommand\makeLineNumber {\hss\thelinenumber\ \hspace{6mm} \rlap{\hskip\textwidth\ \hspace{6.5mm}\thelinenumber}}
% \linenumbers
\pagestyle{headings}
\mainmatter
\def\ECCV16SubNumber{***}  % Insert your submission number here

\title{Kernel Square-Loss Exemplar Machines For Image Retrieval} % Replace with your title

\titlerunning{ECCV-16 submission ID \ECCV16SubNumber}

\authorrunning{ECCV-16 submission ID \ECCV16SubNumber}

\author{Anonymous ECCV submission}
\institute{Paper ID \ECCV16SubNumber}


\maketitle
%\thispagestyle{empty}

%%%%%%%%% ABSTRACT
\begin{abstract}
   In this paper, we propose an improvement to the pipeline of the feature encoder based on the exemplar SVM, first proposed by Zepeda et al.
   First we show that by replacing the hinge loss by the square-loss in the cost function of an Exemplar SVM we obtain similar results on image retrieval in a fraction of the execution time. We call this method Square-loss Exemplar Machine, or SLEM.
   Secondly, we introduce a efficient kernel implementation for SLEM. This implementation scales well for image
\end{abstract}

\input{sec_intro.tex}

\input{sec_prior.tex}

\input{sec_SLEM.tex}

\input{sec_kernel_methods.tex}

\input{sec_eff_imp.tex}

\input{sec_eval.tex}

\bibliographystyle{ieee} 
\bibliography{sup}
\end{document}

%