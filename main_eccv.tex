\documentclass[runningheads]{llncs}

%\usepackage{times}
%\usepackage{epsfig}
\usepackage{graphicx}
\usepackage{caption}
%\usepackage{subfigure}
\usepackage{subcaption}
\usepackage{amsmath}
\usepackage{amssymb}
\usepackage{ruler}
\usepackage{color}
\usepackage[width=122mm,left=12mm,paperwidth=146mm,height=193mm,top=12mm,paperheight=217mm]{geometry}
%\newtheorem{proposition}{Proposition}
%\newtheorem{lemma}{Lemma}
%\newtheorem{proof}{Proof}
\newcommand{\RR}{\mathbb R}
\newcommand{\NNN}{\mathcal N}
\newcommand{\sumi}{\displaystyle{\sum_{i=1}^n}}

\begin{document}
% \renewcommand\thelinenumber{\color[rgb]{0.2,0.5,0.8}\normalfont\sffamily\scriptsize\arabic{linenumber}\color[rgb]{0,0,0}}
% \renewcommand\makeLineNumber {\hss\thelinenumber\ \hspace{6mm} \rlap{\hskip\textwidth\ \hspace{6.5mm}\thelinenumber}}
% \linenumbers
\pagestyle{headings}
\mainmatter
\def\ECCV16SubNumber{***}  % Insert your submission number here

\title{Kernel Square-Loss Exemplar Machines For Image Retrieval} % Replace with your title

\titlerunning{ECCV-16 submission ID \ECCV16SubNumber}

\authorrunning{ECCV-16 submission ID \ECCV16SubNumber}

\author{Anonymous ECCV submission}
\institute{Paper ID \ECCV16SubNumber}


\maketitle
%\thispagestyle{empty}

%%%%%%%%% ABSTRACT
\begin{abstract}
   In this paper, we propose an improvement to the pipeline of the feature encoder based on the exemplar SVM, first proposed by Zepeda et al.
   First we show that by replacing the hinge loss by the square-loss in the cost function of an Exemplar SVM we obtain similar results on image retrieval in a fraction of the execution time. We call this method Square-loss Exemplar Machine, or SLEM.
   Secondly, we introduce a efficient kernel implementation for SLEM. This implementation scales well for image
\end{abstract}

\section{Introduction}
The exemplar SVM (E-SVM) was first introduced by Malisiewicz et al. in \cite{Efros11} as a conceptually simple framework for object detection and image classification where the training set has a small ratio of positive/negative examples. 
At training time, many SVMs are learned from a large pool of negative against a single positive (so called exemplar). 
At test time, the scores of the test images for each classifier are fitted by a logistic regression. 
The final score of an image is then a non-linear combination of scores from multiples exemplar SVMs.

They have also been used in \cite{Efros12} in image retrieval tasks, using the classifier score to rank matching candidates.
However this transfer of information from classification to retrieval is severely limited. 
Indeed, the purpose of a support vector machine is to separate positive and negative samples, i.e., predict discrete labels. 
The distance between a negative sample and the classification hyperplane has no value as a measure of has \emph{a priori} no value of continuous matching score.

Zepeda et al. address this problem in \cite{ZePe15} by firstly, performing an E-SVM to each image in a dataset instead of only for the query images and secondly, comparing its classifiers distance to the query's classifier instead of comparing scores. 
These modifications guarantee we are ranking distances between two points instead of classification scores.
Therefore \cite{ZePe15} refers to E-SVMs as features encoders: a pipeline that takes an image representation as input and returns an improved image representation. This type of feature enhancing is more akin to methods such as whitening, PCA and LDA.

This paper introduces the square-loss Exemplar machine (SLEM), which consists of optimizing the same cost function of a regular E-SVM where the hinge loss is replaced by the square loss. 
The square loss version has the advantage of having a much more efficient optimization. Indeed, the minimization of its cost function is solved by a linear system and can be done for all exemplar simultaneously, whereas the regular E-SVM cost function is generally minimized one exemplar at a time, normally by stochastic gradient descent \cite{bottou10}.
Also, for the machine learning tasks of binary classification, both regular SVM and least-squares SVM have similar performances \cite{YeXi07}.
%Also, for the machine learning tasks of binary classification, both regular SVM and least-squares SVM have the same performance if positive and negative samples are separable and the pool of samples is linearly independent \cite{YeXi07}. 
We also introduce a kernelized version of SLEM, efficiently implemented, that gives better results and superior scalability for large-scale image retrieval problems.


\section{Prior work}
\emph{\color{red} paragraph about image search}

The image retrieval litterature 

%where the query image is the positive example and the rank of an search result image is the score of the score of this image in the classifier obtained by the E-SVM. 

The application of kernel methods for SVM has an extensive bibliography. \emph{\color{red} so extensive that I do not know yet what to put here...}

\section{Square loss exemplar machine}\label{lsesvm}
In this section, we revisit the exemplar SVM model as presented by \cite{Efros11} and introduce its square loss version.
\subsection{Exemplar SVMs} \label{esvm}
Given features in $\RR^d$ at training time, one positive example $x_0$ in $_RR^d$ and a set of negative examples $X = [x_1, x_2,...,x_n]$, in $\RR^{d\times n}$each column of $X$ representing one example by a vector in $\RR^d$. 
Any feature vector representation can be used.
We are also given a loss function $l:\{-1, 1\}\times \RR\rightarrow\RR^+$. Learning an E-SVM from these examples amounts to minimizing the function
\begin{equation}
J(\omega, \nu) = \theta \ l(1, \omega^Tx_0+\nu) +\dfrac{1}{n}\sumi l(-1, \omega^T x_i+\nu)+\dfrac{\lambda}{2}|\!|\omega|\!|^2, \label{eq:first}
\end{equation}
w.r.t. $\omega$ in $\RR^d$ and $\nu$ in $\RR$.

In Equation (\ref{eq:first}), $\lambda$ and $\theta$ are respectively a regularization parameter on $\omega$ and a positive scalar adjusting the weight of the positive exemplar.
%\footnote{We could also acknowledge a parameter $\theta$ other than $\frac{1}{n}$ as a regularization to the error of each negative example. But this parameters seems to be less important to cross-validate. Also, setting $\theta=\frac{1}{n}$ simplify Equation (\ref{omega:solution}) and allows the use of Woodbury identity.} 

The  exemplar SVM of $x_0$ with respect to $\NNN$ is the classifier $\omega^\star$ that minimizes the function $J$:
\begin{equation}
\omega^\star(x_0, X), \nu^\star(x_0, X) = \underset{\omega\in\RR^d, \ \nu\in\RR}{argmin} \ J(\omega, \nu). \label{omega:first}
\end{equation}
To shorten the notation, we refer to these solutions as $\omega^\star$ and $\nu^\star$ when their arguments are implicit.\\
Applications of E-SVM use the hinge loss function, which guarantees that $J$ is a convex function. The solution of Equation (\ref{omega:first}) can be thus found by stochastic gradient descent.

\subsection{Square loss function}\label{SLEM}
Now, let us study the same learning problem for the square-loss function $l(y,\hat{y}) = \frac{1}{2}(y-\hat{y})^2$. As for the hinge loss, Equation (\ref{eq:first}) is a convex problem. 
However, differently from the hinge loss, Equation (\ref{eq:first}) is a ridgr regression problem, whose solution is expressed in a close form:

\begin{align}
\begin{cases}
\vspace{3 mm}
\nu^\star &= \dfrac{\theta-1}{\theta+1}-\dfrac{1}{\theta+1}(\theta x_0+\mu)^T\omega^\star,\\
\vspace{3 mm}
\omega^\star &= \dfrac{2\theta}{\theta+1}U^{-1}(x_0-\mu), \ where \\
\end{cases}
\label{omega:solution}
\end{align}

\begin{align}
where \ \mu = \frac{1}{n}\sum_{i=1}^n x_i, \ \ U = \dfrac{1}{n}XX^T-\mu\mu^T+\dfrac{\theta}{\theta+1}(x_0-\mu)(x_0-\mu)^T+\lambda\textbf{Id}_d.
\end{align}
Equation (\ref{omega:solution}) shows how to solve (\ref{eq:first}) in a closed form to obtain the SLEM vector $\omega^\star(x_0,X)$ of $x_0$. 
Replacing the hinge loss by the square loss offers a more compact solution, but not necessarily more efficient.

\subsection{Woodbury identity and matching scores}
Let us define $A = \frac{1}{n}XX^T-\mu\mu^T +\lambda\textbf{Id}_d$ and assume $A^{-1}$ known. 
%$A$ is the covariance matrix of the negative samples $X$\footnote{added of a small term proportional to the identity matrix to insure $A$ is positive-definite.}. Let us assume its inverse $A^{-1}$ is known. 
We know that $U = A + \dfrac{\theta}{\theta+1}\delta\delta^T$, where $\delta=x_0-\mu$. The Woodbury identity \cite{woodbury} gives us
\begin{equation}
U^{-1} = A^{-1} -\dfrac{\theta}{\theta\delta^TA^{-1}\delta+ \theta+1}A^{-1}\delta^T\delta A^{-1}. \label{invU}
\end{equation}

Replacing (\ref{invU}) in the original equation
\begin{align}
\omega^\star &= \dfrac{2\theta}{\theta+1}U^{-1}\delta \\
&= A^{-1}\delta - \dfrac{\theta}{\theta\delta^TA^{-1}\delta+ \theta+1} A^{-1}\delta (\delta^TA^{-1}\delta)\\
&= \dfrac{2\theta}{\theta\delta^TA^{-1}\delta+ \theta+1} A^{-1}\delta.\label{Wood:omega}
\end{align}

The first observation we make from Equation (\ref{Wood:omega}) is the positive sample weight $\theta$ does not influence the direction of the optimal vector $\omega^\star$, only its norm. This means that according to the matching score criteria defined by \cite{ZePe15}, $\theta$ does not influence the matching score of the SLEM vectors of two different images.

Indeed, if $\omega$ and $\omega'$ are the $d$-dimensional SLEM vectors of $x$ and $x'$, respectively, we can denote $s$ the matching score scalar function defined in $\RR^d\times \RR^d$, which is given by
\begin{equation}
s(\omega, \omega') = \dfrac{1}{|\omega||\omega'|}\langle\omega,\omega'\rangle = (x-\mu)^TA^{-2}(x'-\mu).\label{match:score}
\end{equation}

Equation (\ref{match:score}) is independent from the value of $\theta$. This means that the weight of the positive sample can be fixed at any positive value without changing matching scores. This sets the SLEM appart from E-SVM, where in this parameter had to be cross validated \cite{Efros11, ZePe15}.
\subsection{LDA and SLEM}

Let us now reanalyse the SLEM problem and suppose that we have multiple positive samples. It can be shown that in this case, the corresponding linear classifier of Equation (\ref{eq:first}) for the square-loss is also given by
Equation (\ref{omega:solution}), where $x_0$ denotes this time the center of mass
of the positive samples, {\em if} these samples have the {\em same} covariance matrix $\Sigma$ of the negative samples $X$.
 
This equal-covariance assumption is of course quite restrictive, and
probably unrealistic in general. It is interesting to note, however,
that this is exactly the assumption made by linear discriminant
analysis. As shown in~\cite{Hastie2009} for example, LDA can be seen as a
(non-regularized) linear classifier
with decision function $w'\cdot z+ b'$, where $z$ is a sample in
$\RR^d$, and

\begin{equation}
\left\{\begin{array}{l}
\displaystyle w'=\Sigma^{-1}(x_0-\mu),\\
\displaystyle b'=-\frac{1}{2}(x_0+\mu)^T w',
\end{array}\right.
\label{eq:lda}
\end{equation}
 
Note that when $\lambda=0$ (no regularization),
\begin{align}
\Sigma w & = Uw -\frac{1}{2}[(x_0-\mu)^T w] (x_0-\mu) =
\left[1-\frac{1}{2}(x_0-\mu)^T w\right](x_0-\mu) \\
&=\left[1-\frac{1}{2}(x_0-\mu)^T w\right]\Sigma w'
\end{align}
thus $w$ and $w'$ have the same direction. In other words, a
SLEM is a generalized version of LDA when the
regularizer parameter $\lambda $ is zero.
This observation has been equally made by \cite{Koba15}.

Many interesting properties of LDA has been rediscovered classification tasks \cite{GMPD12,HMR12} and, more recently, for image retrieval with non-embedded image encoders \cite{babenko15}.


\section{Kernel methods for SLEM}
Let us recall a few basic facts about kernel methods for supervised
classification. We consider a reproducing kernel Hilbert space (RKHS)
$H$ formed by real functions over some set
$X$, and denote by $k$ the corresponding reproducing kernel.  We
address the following learning problem over $H\times\RR$:
\begin{equation}
\min_{h\in H,\nu\in\RR}
\sum_{i=1}^n l(y_i,h(x_i)+\nu) + \frac{\lambda}{2}|\! |h|\!|^2,
\label{eq:kernel}
\end{equation}  
where the pairs $(x_i,y_i)$, $i$ in $\{1,\ldots,n\}$ in $X\times \{-1,1\}$ are training samples, and $l: \{-1,1\}\times \RR\rightarrow\RR^+$ is some arbitrary loss function. 
By definition of a reproducing kernel,
Equation (\ref{eq:kernel}) can be rewritten as
\begin{equation}
\min_{h\in H,\nu\in\RR}
\sum_{i=1}^n l(y_i,\langle \varphi(x_i),h\rangle+\nu) +
\dfrac{\lambda}{2}|\!|h|\!|^2,
\label{ker:aff}
\end{equation} 
where $\varphi$ is the {\em feature map} over $X$ associated with the
kernel $k$ (which may not admit a known explicit form) and $\langle \cdot, \cdot \rangle$ is the inner product of $H$. We dub problems with the general form of (\ref{ker:aff}) {\em affine}
supervised learning problems since, given some fixed element $h$ of
$H$ and some scalar $\nu$, $\langle h,h'\rangle+b$ is an affine function of $h'$,
whose zero set defines an affine hyperplane of $H$ considered itself
as an affine space.

Let $K$ denote the kernel matrix with entries $k_{ij}=\langle\varphi(x_i),
\varphi(x_j)\rangle$ and rows $k_i^T=[k_{i1}, k_{i2},...,k_{in}]$, $i$ in $\{1,\ldots,n\}$.  We assume from now on that $l$ is convex and continuous. under this assumption, Equation (\ref{ker:aff}) admits an equivalent version
\begin{align}
\min_{\alpha\in\RR^{n}, \ \nu\in\RR} \left(\dfrac{1}{n}\sumi l(y_i, k_i^T\alpha+\nu)  +\dfrac{\lambda}{2}\alpha^TK\alpha\right), \label{ker:first}
\end{align}
and any solution $(\alpha^\star,\nu^\star)$ to (\ref{ker:first})
provides
a solution $(h^\star,\nu^\star)$ to (\ref{ker:aff}) with
$h^\star=\sum_{i=1}^n \alpha_i^\star\varphi(x_i)+\nu^\star$. This result follows from the representer theorem
~\cite{SHS01,Wahba90}.


%where $K$ is a $(n+1)\times (n+1)$  matrix and $k_i$ is its $(i+1)$-th column matrix for $i$ in $\{0, 1, ..., n\}$.\\

Let us assume from now on that $K$ is a semidefinite positive matrix
%\footnote{\textit{i.e.} for any $z$ in $\RR^n$, $z^TKz\ge 0$. This is equivalent to all eigenvalues of $K$ being non-negative.} 
of rank $r$. If $B$ is the $n\times r$ incomplete Cholesky decomposition of $K$, \textit{i.e.} $K=BB^T$. The kernelized problem can be thus rewritten as 
\begin{equation}
\min_{\beta\in\RR^r,\nu\in\RR} \left( \dfrac{1}{n}\sumi l(y_i, b_i^T\beta+\nu)+\dfrac{\lambda}{2}|\!|\beta|\!|^2\right), \label{beta:first}
\end{equation}
where $b_i^T$ denotes the $i$-th row of $B$.

\subsection{Adding a positive exemplar}
Let us reconsider the exemplar problem, with one positive example $x_0$ and $n$ negative examples $X$. Here $K$ is the kernel matrix in $\RR^{n\times n}$ of the negative samples $X$. 
Let us also assume its factorization $K=BB^T$ and its pseudoinverse $B^\dagger$ are given. We further analyse how to compute $B$ in Subsection~\ref{low-rank}.

We denote $K'$ the augmented kernel matrix obtained by adding the positive samples $x_0$. Such matrix can be written as
\begin{equation}
K = \begin{bmatrix}
k_{00} & k_0^T\\
k_0 & K
\end{bmatrix},
\end{equation}
where $k_{00}=\langle \varphi(x_0),\varphi(x_0)\rangle$ is a scalar and $k_0= [\langle \varphi(x_0),\varphi(x_0)\rangle]_{1\le i\le n}$ is a vector in $\RR^n$. 
The following lemma show that an $(n+1)\times (r+1)$ factorization of $K'$ can be computed as well.

\begin{figure}[!h]
\centering
\includegraphics[width=0.45\textwidth]{imagesInProj.pdf}
\caption{Figure to be completed.}
\label{proj}
\end{figure}

\begin{lemma} The augmented kernel matrix $K'$ can be factorized as
\begin{align}
K'&= B'B'^T,\quad\text{where}\quad
B'=\begin{bmatrix}
u & v^T\\0 & B
\end{bmatrix}\\
v&=B^\dagger k_0,\,\, u=\sqrt{k_{00}-||v||^2}.\label{eq:lemma1}
\end{align}
\end{lemma}\label{lemma1}
\begin{proof}
For $B'$ defined by (\ref{eq:lemma1}), we have that 
\begin{equation}
B'B'^T = \begin{bmatrix} u+|\!|v|\!|^2 & v^TB^T\\ 
Bv& BB^T\end{bmatrix}.
\end{equation}
Since $K'$ is positive semidefinite, $k_0$ must lie in the column space $\mathcal{B}$ of $B$.
%\footnote{
Indeed, if we suppose $k_0$ does not belong to $\mathcal{B}$, then it can be decomposed uniquely as $k_0=s+t$, $s\in\mathcal{B}$ and  $t\in\mathcal{B}^\perp$, with $t\ne 0$. In one hand, $K'$ being semidefinite positive implies that $[1, -at^T]K'[1; -at]=k_{00}-2a||t||^2\ge 0$ for all real value $a$. In the other hand, for $a$ large enough, $k_{00}-a||t||^2\le 0$, which is a contradiction.
%} 
Hence $v=B^\dagger k_0$ is an exact solution of $Bv=k_0$. The fact that $k_{00}-|\!|v|\!|^2$ is non-negative comes from the fact that the Schur complement $K-|\!|k_0|\!|^2/k_{00}$ of $k_{00}$ in $K'$ is itself positive semidefinite.
%\footnote{
Indeed, since the matrix $k_{00}K-k_0k_0^T=B(k_{00}\textbf{Id}_r-vv^T)B^T$ is positive semidefinite and $B$ has rank $r$, $k_{00}\textbf{Id}_r-vv^T$ is also positive semidefinite. Thus $v^T(k_{00}\textbf{Id}_r-vv^T)v = ||v||^2(k_{00}-||v||^2)\ge 0$.
%}
\end{proof}
This lemma allows us to kernelize Equation (\ref{omega:first}) for any positive exemplar $x_0$ for a fixed set of negatives $X$:
%$\beta^\star(x_0,X), \nu^\star(x_0,X) = \underset{\beta\in\RR^{r+1}, \nu\in\RR}{argmin}J'(\beta, \nu)$, for
$\beta^\star(x_0,X), \nu^\star(x_0,X) = argmin \ J'(\beta, \nu)$, where
\begin{align}
J'(\beta, \nu)=
 \theta\ l(1, b_0'^T\beta+\nu) +\dfrac{1}{n}\sumi l(-1,b_i'^T\beta+\nu)
+\dfrac{\lambda}{2}|\!|\beta|\!|^2,\label{beta:final}
\end{align}
and $b_i'^T$ is the $(i+1)$-th row of $B'$, $i$ in $\{0,1,...,n\}$. In particular, $b_0'=[u; \ v]$ and, for $i>0$, $b_i'=[0; \ b_i]$. $\beta^\star$ and $\nu^\star$ can be computed just as before by Equation (\ref{omega:solution}), replacing $x_0$ by $b_0'$, $X$ by the $(r+1)\times n$ matrix $Q$ of columns $b_1', b_2',...,b_n'$ and $\mu$ by $\mu' = \frac{1}{n}\sum_{i=1}^n b_i'$.


\subsection{Matching score}
Once the optimal parameters $(\beta, \nu)$ from (\ref{beta:final}) and the coordinates $u$, $v$ of $b_0'$ from (\ref{eq:lemma1}) have been found\footnote{We drop the "$\star$" in this subsection to avoid clutter up the notation.}, they can be used directly for measuring similarity between matching images. 
Indeed, the corresponding vector $\alpha$ (or, more correctly, \textit{a} corresponding value of dimension $n+1$) can be computed by $\alpha = P'\beta$, where $P'=B'(B'^TB')^{-1}$  is the pseudoinverse of $B'^T$. This identity can be written as 
\begin{equation}
\begin{bmatrix} \alpha_0 \\ \hat{\alpha} \end{bmatrix} = \begin{bmatrix} \frac{1}{u} & 0^T \\-\frac{1}{u}Pv & P  \end{bmatrix} \begin{bmatrix}\beta_0 \\ \hat{\beta} \end{bmatrix},
\end{equation}
for $P$ pseudoinverse of $B^T$, $\alpha_0$ and $\beta_0$ scalars.
Inspired by the matching score of \cite{ZePe15}, the matching score $s(h,h')$ of $h$ and any $h'=\alpha_0'\varphi(x_0')+\sum_{i=1}^n \alpha_i'\varphi (x_i)+\nu'$ in $H$, the representation of another image $x_0'$, is given by 
\begin{align}
 s(h,h') &= \langle h^\star, h'\rangle= \hat{\alpha}^{T} K\hat{\alpha}'+\alpha_0k(X, x_0)^T\hat{\alpha}'+\alpha_0'k(X, x_0')^T\hat{\alpha}+\alpha_0\alpha_0'k(x_0,x_0')\\
 &=
\hat{\beta}^T\hat{\beta}'+\frac{\beta_0\beta_0'}{uu'}(k(x_0,x_0')-v^Tv').
\end{align}





\section{Efficient implementation}\label{eff_imp}
When compared with the problem of (\ref{omega:first}), one drawback of a kernelized approach of (\ref{ker:first}) is that the dimension of our problem grows with the size $n$ of the negative samples.
We can compute the Cholesky factor $B$ of $K$ in $O(nr^2)$ time and $O(nr)$ storage, as will be shown in subsection~\ref{low-rank}. 
For each positive exemplar, solving Equation (\ref{beta:final}) amounts to computing $B'$ from Lemma 1, and solving a linear system in $U$, which is a $(r+1)\times (r+1)$ matrix. The first of these is calculated in time $O(nr)$ and the second, $O(r^3)$.

Two ideas allow us to solve the kernelized problem linearly in $n$: low-rank decomposition of the kernel matrix $K$, to diminish the value of $r$ and the Woodbury identity, so we only have to solve one linear system for all positives. 

\subsection{Low-rank decomposition} \label{low-rank}
In this section we start by revisiting the Cholesky algorithm of the construction of $B$, as presented in ~\cite{BaJo05,FiSc01}. 
We denote, for any list of indexes $M$, $N$ in $\{1,2,...,n\}$, $K(M,N)$ the submatrix of $K$ composed of the rows indexed by $M$ and columns indexed by $N$ and $K(M,:)$ the submatrix of $K$ composed by rows indexed by $M$ and all columns.
Let us assume a sequence of indexes $I_r=\{i_1,i_2,\dots ,i_r\}$ from $\{1,2,\dots, n\}$ is known. 
We also denote, for $t$ in $\{1,2,...,r\}$, $I_t=\{i_1,...,i_t\}$ the subset of the $t$ first indexes of $I_r$and $J_t=\{1,2,...,n\}/ I_t$ the complement of the $t$ first indexes.
$I_r$ is the set of pivots, chosen greedily such that $i_t$ minimizes the error approximation between $B_tB_t^T$ and $K$, where $B_t=B(I_t,:)$. 

Given $I_t$, we can calculate the $t$-th column, $t=1,2,...,r$, of $B$ iteratively:
\begin{align}
\begin{cases}
\vspace{3 mm}
B(i_t, t) = \left(K(i_t,i_t)-\sum_{m=1}^{t-1} B(i_{m},m)  \right)^{\frac{1}{2}},\\
\vspace{3 mm}
B(I_{t-1},t) = 0,\\
B(J_t, t) = \frac{1}{B(i_t,t)}(K(J_t,i_t)\\
\ \ \ \ \ \ \ \ \ \ \ \ \ \ \ \ \ \ \ \ \ \ \ \ \ \ \  \ \ \ \  -\sum_{m=1}^{t-1}B(J_t,m)B(i_t,m)
).\end{cases}\label{icd:algo}
\end{align}
The time complexity of the $t$-th interaction is $O(tn)$, thus making the full algorithm $O(nr^2)$ in time. Since $B$ is $n\times r$, storage complexity is $O(nr)$. In particular, \cite{BaJo05} shows us that, for $t$ in $\{1,2,...,r\}$, the error matrix $K-B_tB_t^T$ is a $(n-t)\times (n-t)$ block matrix which is the Schur complement of $K(I_t,I_t)$ in $K$ and zero elsewhere.

For many kernels such as the Gaussian, $K$ can be a full rank matrix. Its Cholesky decomposition is hence a $O(n^3)$ operation in time, which is a prohibitive cost for large negative sets $X$. We propose stoping the algorithm described in (\ref{icd:algo}) after a given number $r'<r$ of steps, and use $B_{r'}$ as an approximation of $B$. We also need to know only the $r'$ first pivots, which we call $I$ from now on. Conversely, we note $J=J_{r'}$. The construction of $B'$ from $B_{r'}$ can be adapted as presented in the following lemma:
%\begin{proposition}
%There is an unique $n\times n$ matrix $L$ such that
%\begin{itemize}
%\item L is symmetric and positive semidefinite,
%\item L(:,I) = K(:,I),
%\item the column space of $L$ is equal to the column space of $L(:,I)$.
%\end{itemize}
%L is such that 
%\begin{equation}
%L([I \ J],[I\ J]) = \begin{bmatrix}
%K(I,I) && K(J,I)^T\\ K(J,I) && K(J,I)K(I,I)^{-1}K(J,I)^T
%\end{bmatrix}.
%\end{equation}
%\end{proposition}
%From \textbf{Proposition 1}, $L$ is an unique approximation of $K$ for a given pair $I$ and $J$. The algorithm from (\ref{icd:algo}) construct $B$ such that $L=BB^T$.
\begin{lemma}
The augmented kernel matrix $K'$ can be factorized as
\begin{align}
K'&\approx B'B'^T,\quad\text{where}\quad
B'=\begin{bmatrix} u & v^T\\w & B_{r'} \end{bmatrix},\quad
v=B_I^{-1} k_0(I),\\u&=\sqrt{k_{00}-||v||^2},\,\, w(I) = 0,\,\, w(J)=\dfrac{1}{u}(k_0(J)-B_Jv)
\end{align}
in time $O(nr')$ and storage $O(n)$. Also, the approximation error of $K'$ and $B'B'^T$ is the same of $K$ and $B_{r'}B_{r'}^T$. Here, $B_I$ and $B_J$ denote $B(I,I)$ and $B(J,I)$, respectively.
\end{lemma}
The proof of this lemma is similar to the previous lemma's proof, only adding the $\RR^n$ vector $w$ whose entries are calculated by Equation (\ref{icd:algo}).
%\begin{proof}
%As before, we have that
%\begin{equation}
%B'B'^T = \begin{bmatrix} u+|\!|v|\!|^2 & v^TB_{r'}^T+uw^T\\ 
%B_{r'}v+uw& BB^T+ww^T\end{bmatrix}.
%\end{equation}
%If $E = K'-B'B'^T$,
%Algorithm (\ref{icd:algo}) is such that the rank of $B_{r'}$ is $r'$, for all $r'<r$.
%\end{proof}
\subsection{Woodbury identity}
Let us call $A=\frac{1}{n}QQ^T-\mu'\mu'^T+\lambda\textbf{Id}_{r+1}$. We assume its inverse $A^{-1}$ known. From (\ref{omega:solution}) for the kernelized version, we know that $U=A+\frac{\theta}{\theta+1}\delta\delta^T$, for $\delta=b_0'-\mu'$. The Woodbury identity gives us 
\begin{equation}
U^{-1}=A^{-1} -\dfrac{\theta}{\theta\delta^TA^{-1}\delta+ \theta+1}A^{-1}\delta^T\delta A^{-1}.\label{wood1}
\end{equation}
Replacing (\ref{wood1}) in the original equation 
\begin{equation}
\beta^\star = \dfrac{2\theta}{\theta+1}U^{-1}\delta = \dfrac{2\theta}{\theta\delta^TA^{-1}\delta+ \theta+1}A^{-1}\delta,\label{beta:fromwood}
\end{equation}
which gives the solution $\beta^\star$ dependent only $A^{-1}$, $\theta$, $\mu'$ and $b_0'$. 
We can thus solve only one system for $A$ for \textit{all} positives, instead of one in $U$ for \textit{each} positive.

If we decompose $K$ at rank $r$, $A$ is constant for all positives, since $Q = [\textbf{0}, B]$ and $\mu'=[0; \mu]$.\footnote{Here, $\mu$ is the center of mass of the rows of $B$, not the columns of $X$ as initially presented in section \ref{lsesvm}.} 
But for a low-rank decomposition, $Q=[w, B_{r'}]$ and $\mu' = [\bar{w}; \mu]$. If so, $A$ is written as 
\begin{equation}
A = \begin{bmatrix}
a_{00} & a_0^T\\
a_0^T & G
\end{bmatrix},
\end{equation}
where $a_{00}=\dfrac{1}{n}w^Tw-\bar{w}^2+\lambda$, $a_0 = \dfrac{1}{n}B_{r'}^Tw-\bar{w}\mu$ and $G=\dfrac{1}{n}B_{r'}^TB_{r'}-\mu\mu^T+\lambda\text{Id}_r$.

$G$ depends only on $B$ and can be calculated in time $O(nr'^2)$ and storage $O(r'^2)$. 
Its inverse $G^{-1}$ is computed at cost $O(r'^3)$.
The Woodbury identity, again, allows the computation of $A^{-1}$ from $G^{-1}$ at cost $O(r^2)$:
\begin{equation}
A^{-1} = \begin{bmatrix}\gamma & -\gamma a_0^TG^{-1}\\ -\gamma G^{-1}a_0 & G^{-1}+\gamma G^{-1}a_0a_0^TG^{-1}\end{bmatrix},\label{wood2}
\end{equation}
where $\gamma = \left(a_{00}-a_0^TG^{-1}a_0\right)^{-1}$. Finally, replacing (\ref{wood2}) in (\ref{beta:fromwood}), we can obtain $\beta^\star$ by solving the linear problem $A\beta^\star = \delta$. 
Both $A$ and $\delta$ are dependent on $u$, $v$ and $w$, which are the computed for each exemplar as described by Lemma 2.

\input{sec_eval.tex}

\bibliographystyle{ieee} 
\bibliography{sup}
\end{document}

%